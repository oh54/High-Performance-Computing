The assignment seeks to solve the Poisson equation in two dimensions and the specific stated steady state heat problem of the heat distribution in a square room with a known boundary conditions. Taking this information, the goal of this assignment is to implement sequential versions of two iterative PDE solvers, namely the Jacobi method and the Gauss-Seidel method, which are both multi grid smoothers, that can be used to find the solution to the Poisson problem. The Jacobi method is parallelized using openMP and the scalability is compared to the parallelization of the calculation of the Mandelbrot set. All of the mentioned implimentations are then analyzed for different parameters and we seek to determine the prospects and limitations of the different approaches. Finally a method for parallelizing the Gauss-Seidel method is discussed but not implemented.

\subsection{Computing Specifications}
Unless otherwise stated a lot of the specifications are fixed throughout the report. The code is written in C and compiled with the Sun Studio compiler. All the sequential experiments are run on the hpcintro queue on a single node with one processors per node (ppn), which is specified in the submit.sh script. The details of the parallel experiments will be specified on the go. All the nodes on the cluster are identical, and the full specifications are given in table \ref{tab:spec}.
\begin{table}[h!]
\centering
\begin{tabular}{l l}
Architecture:     &     x86\_64 \\ \hline
CPU op-mode(s):    &    32-bit, 64-bit\\ \hline
Byte Order:         &   Little Endian\\ \hline
CPU(s):             &   20\\ \hline
On-line CPU(s) list:&   0-19\\ \hline
Thread(s) per core: &   1\\ \hline
Core(s) per socket: &   10\\ \hline
Socket(s):          &   2\\ \hline
NUMA node(s):       &   2\\ \hline
Vendor ID:          &   GenuineIntel\\ \hline
CPU family:         &   6\\ \hline
Model:              &   63\\ \hline
Stepping:           &   2\\ \hline
CPU MHz:            &   2601.000\\ \hline
BogoMIPS:           &   5187.68\\ \hline
Virtualization:     &   VT-x\\ \hline
L1d cache:          &   32kB\\ \hline
L1i cache:          &   32kB\\ \hline
L2 cache:           &   256kB\\ \hline
L3 cache:           &   25600kB\\ \hline
NUMA node0 CPU(s):  &   0-9\\ \hline
NUMA node1 CPU(s):  &   10-19\\
\end{tabular}
\caption{Table of the full machine specifications.}
\label{tab:spec}
\end{table}
