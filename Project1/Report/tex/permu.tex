\subsection{The Permutations}

Since the algorithm consists of three nested loops iterating over the three dimensions m, n and k, it is clear through basic combinatorics, that there is 3! = 6 possible ways to order the loops:\\
mnk - mkn - kmn - knm - nmk - nkm \\
The 6 permutations are implemented as separate functions in the library with names matmult\_NNN. The functions are obviously very similar to the already described nat function, but the function are however displayed below:

\begin{lstlisting}[language=C++, caption=lib]
void matmult_mkn(int m, int n, int k, double ** A, double ** B, double ** C){
	for(int i = 0; i < m;i++){
		for(int j = 0; j < n;j++){
			C[i][j] = 0;		
		}
	}


	for(int i = 0; i < m;i++){
		for(int l = 0; l < k;l++){
			for(int j = 0; j < n;j++){
				C[i][j] += A[i][l]*B[l][j];
			}
		}
	}
}
\end{lstlisting}

Here, the C matrix is defined in an earlier loop. The mkn permutation is here shown, since it will also turn out to be an important permutation throughout this report.